
%%% Local Variables:
%%% mode: latex
%%% TeX-master: t
%%% End:
\section{数据在业务中的应用案例}

\subsection{常规交通指标分析}

\begin{frame}[t]{\subsecname}
  \begin{itemize}
     \item<1->出租车
        \begin{enumerate}
          \item 客流量分析(乘次、产生量和吸引量统计)
          \item 出行分析(OD矩阵、OD行程时间和出行距离)
          \item 运营分析(空驶比例、路段行程车速)
        \end{enumerate}
     \item<2->定点车流
        \begin{enumerate}
          \item 流量统计
          \item 流量时变分析
          \item 流向分析(各关口每日出入关的总流量)
        \end{enumerate}
     \item<3->常规公交
        \begin{enumerate}
          \item 客流量分析(线路、站点、换乘)
          \item 出行分析(站点OD、出行OD)
          \item 运营分析(车速、发车频率、候车时间)
          \item 可达性分析
          \item 站点覆盖范围分析
        \end{enumerate}
     \item<4->轨道
        \begin{enumerate}
          \item 客流量统计
          \item 出行统计(站点OD、出行OD、OD行程时间、出行距离)
          \item 运营统计(线路拥挤度、站点拥挤度)
        \end{enumerate}
  \end{itemize}
\end{frame}

\begin{frame}[t]{\subsecname}
  \begin{itemize}
     \item<1-> 居民出行调查
        \begin{enumerate}
          \item 人口、就业、年龄、收入等分布
          \item 交通方式和交通目的比例
          \item 分交通方式和交通目的的出行量
          \item 通勤交通分析 
        \end{enumerate}
  \end{itemize}
  \begin{itemize}
     \item<2-> 手机
        \begin{enumerate}
          \item 就业和居住人口分布
          \item 不分交通方式和目的的出行量
          \item 通勤交通分析 
        \end{enumerate}
  \end{itemize}

\only<3>{
\begin{badbox}{多种数据融合使用} 
   当遇到同一指标存在多种数据源都可以计算的情况,针对应用场景选择最合适的数据进行计算
\end{badbox}}
\end{frame}

\subsection{都市圈范围分析}

\begin{frame}[t]{\subsecname}
\begin{itemize}
  \item<1-> 数据:2010和2016年居民出行调查
  \item<2-> 分析指标:通勤率=某区域至中心区域通勤人数/该区域常住人口
  \item<3-> 解决问题:深圳市城市边界的空间增长与发展规律
\end{itemize}

\end{frame}

\subsection{区域联系强度分析}

\subsection{货运交通分析}


